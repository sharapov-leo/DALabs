\documentclass[12pt]{article}

\usepackage{fullpage}
\usepackage{multicol,multirow}
\usepackage{tabularx}
\usepackage{ulem}
\usepackage[utf8]{inputenc}
\usepackage[russian]{babel}

\begin{document}

\section*{Лабораторная работа №\,5 по курсу дискрeтного анализа: суффиксные деревья}

Выполнил студент группы 08-308 МАИ \textit{Шарапов Леонид}.

\subsection*{Условие}

\begin{enumerate}
\item Общая постановка задачи.

Необходимо реализовать алгоритм Укконена построения суффиксного дерева за линейное время. Используются строчные буквы латинского алфавита.

\item Вариант 4. 

Линеаризовать циклическую строку, то есть найти минимальный в лексикографическом смысле разрез циклической строки.

\end{enumerate}

\subsection*{Метод решения}

\begin{enumerate}
\item Считать строку
\item Удвоить строку, так как это нужно для варианта задания
\item Составить суффиксное дерево с помощью алгоритма Укконена
\item Начиная с корня, спускаться по самым младшим дочерним узлам, количество распечатанных символов меньше длины строки (не равно)
\item Распечатать строки, соответствующие им
\end{enumerate}

\subsection*{Описание программы}

Весь код хранится в main.cpp

Используемые типы данных
\begin{enumerate}
\item const int
\item class suffixTree - каждый его объект соответствует состоянию класса
\item int
\item int*
\item std::string
\item std::shared\_ptr - для автоматического очищения памяти
\item struct active - хранит состояние, позицию символа и длину подстроки активной точки

Используемые функции
\begin{enumerate}
\item suffixTree::suffixTree(int start, int* end) - конструктор, заполняющий состояние. Дочерние указатели зануляются, суффиксная ссылка ведет на корень, а начальная и конечная позиции в тексте суффикса задаются по переданным параметрам
\item suffixTree::suffixLength() - возвращает длину суффикса
\item suffixTree::addCharacterToTree(int pos) - добавляет в суффиксное дерево символ, стоящий на позиции pos в тексте
\item suffixTree::linearizeTheCyclicString() - выводит минимальный лексикографический разрев заданной строки
\item suffixTree::setSuffixLink() - устанавливает суффиксную ссылку для состояния на корень
\end{enumerate}

\end{enumerate}

\subsection*{Дневник отладки}

\begin{enumerate}
\item wrong answer test01.t - в выводе не было перехода на новую строку
\item runtime error test04.t got signal 6 - неправильное удлинение строк, то есть не было нормальной перестановки указателя со старой памяти на новую и очистки старой памяти, поэтому был выход за границу массива
\item runtime error test09.t got signal 11
\end{enumerate}

\subsection*{Выводы}

С помощью алгоритма Укконена можно быстро построить суффиксное дерево, применяемое для поиска образца в строке, быстрое нахождение последних символов циклических сдвигов строки в преобразовании Барроуза-Веллера, задача о словаре и другие приложения. Очень сложная теория. Основные проблемы были с ней и неправильной работой со строками.

\end{document}

\documentclass[12pt]{article}

\usepackage{fullpage}
\usepackage{multicol,multirow}
\usepackage{tabularx}
\usepackage{ulem}
\usepackage[utf8]{inputenc}
\usepackage[russian]{babel}

\begin{document}

\section*{Лабораторная работа №\,6 по курсу дискрeтного анализа: калькулятор}

Выполнил студент группы 08-308 МАИ \textit{Шарапов Леонид}.

\subsection*{Условие}

Составить программу, использующую программную библиотеку, реализующую простейшие арифметические действия и проверку условий над целыми неотрицательными числами.

\subsection*{Метод решения}

\begin{enumerate}
\item Реализован класс длинного числа с перегрузками требуемых арифметических операций и условий
\item Перегружены операции поточного ввода и вывода
\item Основание системы счисления равняется 10000
\item Считываются длинные числа, а затем требуемая операция
\end{enumerate}

\subsection*{Описание программы}

Весь код находится в main.cpp

Используемые типы данных
\begin{enumerate}
\item std::vector<int> num - хранение длинного числа
\item const int - основание системы счисления и количество нулей в ней
\item int
\item std::string - введенное число
\item char - операция
\item longNumber - длинное число
\end{enumerate}

Используемые функции
\begin{enumerate}
\item main - считывание чисел и операции, а затем выбор действий
\item longNumber::longNumber(int n) - конструктор для считывания длинного числа через int
\item std::istream\& operator>>(std::istream\& in, longNumber\& obj) - перегрузка поточного ввода
\item std::ostream\& operator<<(std::ostream\& out, longNumber obj) - перегрузка поточного вывода
\item longNumber longNumber::operator+(const longNumber\& right) const - перегрузка операции сложения длинных чисел
\item void longNumber::delLeadingZeros() - удаление незначащих нулей из длинного числа после выполнения операции
\item longNumber longNumber::operator-(const longNumber\& right) const - перегрузка операции вычитания длинных чисел
\item longNumber longNumber::operator*(const longNumber\& right) const - перегрузка операции умножения длинных чисел
\item longNumber longNumber::operator/(const longNumber\& right) const - перегрузка операции деления длинных чисел
\item longNumber longNumber::operator\^(const longNumber\& right) const - перегрузка операции возведения в степень длинного числа
\item longNumber binPow(const longNumber\& right) const - рекурсивное быстрое возведение в степень
\item bool longNumber::operator==(const longNumber\& right) const - перегрузка оператора равенства длинных чисел
\item int longNumber::cmp(longNumber\& obj) const - функция сравнения длинных чисел, возвращающая 0, если они равны, иначе -1 или 1, если первое больше второго и наоборот
\end{enumerate}

\subsection*{Дневник отладки}
\begin{enumerate}
\item time limit 10pow.t - для 0 в степени x была долгая работа при больших x > 0. Введено условие, чтобы было быстрее
\item time limit 10pow.t - переделано бинарное возведение в степень из итеративной формы в рекурсивную
\item wrong answer 3minus.t - при вычитании из 10000 единицы l - r + k становилось меньше нуля. Исправлено
\item wrong answer 6pow.t - для 0 в степени 0 должна выводится ошибка
\end{enumerate}

\subsection*{Выводы}

Длинная арифметика используется для вычислений на ПК, потому что типы переменных в них ограничены числом, то есть для избежания переполнения. Фундаментальные математические константы, например, $\pi$ могут быть вычислены с ее помощью. Основными проблемами были долгий поиск теории, работа с ней и техническим заданием. 

\end{document}

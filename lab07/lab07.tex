\documentclass[12pt]{article}

\usepackage{fullpage}
\usepackage{multicol,multirow}
\usepackage{tabularx}
\usepackage{ulem}
\usepackage[utf8]{inputenc}
\usepackage[russian]{babel}

\begin{document}

\section*{Лабораторная работа №\,7 по курсу дискрeтного анализа: динамическое программирование}

Выполнил студент группы 08-308 МАИ \textit{Шарапов Леонид}.

\subsection*{Условие}
\begin{enumerate}
\item Общая постановка задачи.

Используя метод динамического программирования, разработать алгоритм решения задачи, определяемой вариантом, оценить время выполнения алгоритма и объем потребляемой оперативной памяти. Перед выполнением задания необходимо обосновать применимость метода динамического программирования. Разработать программу на языке C или C++, реализующую построенный алгоритм. Формат входных и выходных данных описан в варианте зададния.

\item Вариант 4. Игра с числом

Имеется натуральное число n. За один ход с ним можно произвести следующие действия:

\begin{enumerate}
\item Вычесть единицу
\item Разделить на два
\item Разделить на три
\end{enumerate}

Cтоимостью каждой операции является текущее значение n, а преобразования - суммарная стоимость всех операций в в нем. Необходимо с помощью указанных операций преобразовать число n в единицу, что стоимость преобразования будет наименьшей. Делить можно только нацело.

\end{enumerate}

\subsection*{Метод решения}

\begin{enumerate}
\item Построено дерево решений
\item Найдено оптимальное решение: нисходящий анализ. Для чисел 10 -> 1 считать стоимости выполнения операций для полученных чисел, учитывая, что нужна наименьшая из них
\item При подсчете оценки для каждого числа записывать выполненную операцию
\item От 1 до 10 провести обратные операции, чтобы узнать последовательность из них, при которой получается наименьшая оценка
\item Вывести в обратном порядке полученные операции
\end{enumerate}

\subsection*{Описание программы}

Весь код находится в main.cpp

Используемые типы данных
\begin{enumerate}
\item unsigned long long
\item char
\item struct node
\item std::vector<int>
\item std::string
\end{enumerate}


\subsection*{Дневник отладки}

\begin{enumerate}
\item wrong answer at test01.t - не было перехода на новую строку после вывода ответа
\item wrong answer at test11.t - было переполнение int. Заменено на беззнаковый int
\item wrong answer at test16.t - было переполнение беззнакового int. Заменено на беззнаковый long long
\end{enumerate}

\subsection*{Выводы}

Данная задача встречается в олимпиадном программировании. Основной сложностью было нахождение оптимального решения. Для решения данной задачи можно использовать нисходящий и восходящий анализы. Так как был использован первый из них, то стоимости операций не помещались в int, поэтому был использован тип long long. 

\end{document}
\documentclass[12pt]{article}

\usepackage{fullpage}
\usepackage{multicol,multirow}
\usepackage{tabularx}
\usepackage{ulem}
\usepackage[utf8]{inputenc}
\usepackage[russian]{babel}

\begin{document}

\section*{Лабораторная работа №\,8 по курсу дискрeтного анализа: жадные алгоритмы}

Выполнил студент группы 08-308 МАИ \textit{Шарапов Леонид}.

\subsection*{Условие}

\begin{enumerate}
\item Общая постановка задачи (один абзац).

Разработать жадный алгоритм решения задачи.

Реализовать программу на языке C или C++, соответствующую построенному алгоритму.

\item Вариант задания. 

Заданы N объектов с ограничиениями на расположение вида «A должен находиться перед B». Необходимо найти такой порядок расположения объектов, что все ограничения будут выполняться. Входные данные: на первой строке два числа, N и M, за которыми следует M строк с ограничениями вида «A B» (1 <= A, B <= N ) определяющими относительную последовательность объектов с номерами A и B. Выходные данные: -1 если расположить объекты в соответствии с требованиями невозможно, последовательность номеров объектов в противном случае.

\end{enumerate}

\subsection*{Метод решения}

\begin{enumerate}
\item Считываются количество узлов и ограничений
\item Считываются списки смежности для орграфа
\item Для каждого узла применяется алгоритм Тарьяна
\item Выводится согласованная последовательность узлов
\end{enumerate}

\subsection*{Описание программы}

Весь код находится в main.cpp

Используемые функции
\begin{enumerate}
\item main
\item taryanAlgo
\end{enumerate}

Используемые типы данных
\begin{enumerate}
\item std::vector
\item int
\item char
\item std::istream
\item std::ostream
\end{enumerate}

\subsection*{Выводы}

Алгоритм Тарьяна применяется для поиска компонент сильной связности в орграфе. Основной сложностью был разбор теории. Программирование было несложным

\end{document}
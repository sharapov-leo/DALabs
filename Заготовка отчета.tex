\documentclass[12pt]{article}

\usepackage{fullpage}
\usepackage{multicol,multirow}
\usepackage{tabularx}
\usepackage{ulem}
\usepackage[utf8]{inputenc}
\usepackage[russian]{babel}

% Оригиналный шаблон: http://k806.ru/dalabs/da-report-template-2012.tex

\begin{document}

\section*{Лабораторная работа №\,1 по курсу дискрeтного анализа: сортировка за линейное время}

Выполнил студент группы 08-209 МАИ \textit{Пупкин Василий}.

\subsection*{Условие}

Кратко описывается задача: 
\begin{enumerate}
\item Общая постановка задачи (один абзац).
\item Вариант задания. 
\end{enumerate}

\subsection*{Метод решения}

Общее описание алгоритма решения задачи, архитектуры программы и
т.\,п. Полностью расписывать алгоритмы необязательно, но в общих чертах
описать нужно. Приветствуются ссылки на внешние источники,
использованные при подготовке (книги, интернет-ресурсы). 

\subsection*{Описание программы}

Разделение по файлам, описание основных типов данных и функций. 

\subsection*{Дневник отладки}

Что и когда делали, что не работало, как чинили.
Этот пункт обязательный, если если было сделано несколько
посылок. Кратко опишите суть проблемы и способ ее устранения.

\subsection*{Тест производительности}

Померить время работы кода лабораторной и теста производительности
на разных объемах входных данных. Сравнить результаты. Проверить,
что рост времени работы при увеличении объема входных данных
согласуется с заявленной сложностью.


\subsection*{Недочёты}

Описать те проблемы, которые остались нерешёнными в программе. Если
выявленных недочётов нет, то этот пункт можно пропустить. Если же
задача была принята с оговорками (медленно работает, неверно выполняет
какие-то тесты), то нужно обязательно их упомянуть, привести возможные
причины, указать почему не получилось исправить недочёты. 

\subsection*{Выводы}

Описать область применения реализованного алгоритма. Указать типовые
задачи, решаемые им. Оценить сложность программирования, кратко
описать возникшие проблемы при решении задачи. 

\end{document}